FullFAT is available Free to anyone under the GNU v3 Free Software License. This is intended to allow anyone, private or commercially, to test, proto-type and develop software or devices using FullFAT. This requires that the source-code be made available, along with the source-code of your project showing the integration of FullFAT, and any modifications or patches to FullFAT that you have written.
If you have modified FullFAT in anyway, this must be clearly stated, and marked within the source-code. All patches must be sent to the author, for possible inclusion in a future release of FullFAT.
\newline
\newline
What the license doesn't allow is a product to be sold, without the distribution of the FullFAT source-code, including the code used to integrate it into your project. (FullFAT drivers that you wrote etc).
\newline
\newline
Because in many circumstances, this is undesirable, a commercial license can be obtained from the author. The benefits of such a license include:
\begin{itemize}
\item No need to mention that you used FullFAT in your product.
\item Commercial support for integration, based on an agreed hourly rate.
\item Guaranteed bugfixes for major bugfixes within an agreed timeframe.
\item Guaranteed bugfixes for minor bugfixes within an agreed timeframe.
\item A better version of the source-code including new features, encapsulated data-types, and proper documentation.
\end{itemize}

A number of licensing options can be arranged, its all up for negotiation so get in contact with me to arrange a reasonable price for your specific needs.
\newpage
\subsection{The Open-source License}
\scriptsize
\verbatiminput{pages/license.txt}
